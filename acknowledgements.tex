% -*- coding: utf-8 -*-

%\makeschapterhead{致谢}
\chapter*{致谢}

值此论文完成之际,南开哺我已近七载,于网络实验室的学习也已三年有余。感激之情虽溢于言表,却不知何以为报。唯有尽心尽力完成此文来表达对学校、恩师和朋友们的感谢。

我在实验室学习的三年期间,我的导师张建忠老师一直对我的科研工作进行悉心的指导。每逢遇到瓶颈之时,张老师总会不厌其烦的指引教导我,如果没有张老师在学习上的指导,此文不知会陷于何等的混乱。张老师渊博的专业知识、严谨的治学态度、一丝不苟的工作作风以及于生活上对学生的关怀,都是为人之师表的典范,也时时感染和激励着我,纵使愚钝如我亦能受益良多。“自惭菲薄才,误蒙师长恩。”这三年间,我无时无刻不因自己有限的个人能力而感到遗憾,更待在以后的工作中初有建树之时,方能对张老师的培养感到无愧。

同样感谢徐敬东老师在我科研创作上给予的帮助,徐老师在我发表会议论文的过程中不吝指导,使我于论文创作上初窥门径。徐老师在工作和科研中兢兢业业、以身作则,生活中也对学生关怀备至,都是我奉为学习的榜样,使我受益良多。

感谢张玉老师,张玉老师带我走进流分类的科研领域,为我提供了宝贵的科研资料,于我的科研工作有巨大的帮助。因研究所需出差之际遇,与张玉老师有了更多科研之外的分享,其不管于科研方面还是生活方面严肃认真的性格都是值得我学习的。感谢吴英老师、许昱伟老师、蒲凌君老师,感谢你们在学习和生活上给予我的帮助。

感谢王昌海、于文平、郭彦斌、王景璟、马超、谢玉婷等师兄师姐在学习、生活和就业等方面给予我的帮助。感谢金宇菲、付宁佳、武欢、王青江等同学,感谢与我在学习中一起成长的你们。感谢李洋、尹铭鑫、黄飞龙、付缺、邵祎然、胡永康、蔺慧霞等师弟师妹,感谢你们给实验室带来的新的活力。

研究创作不易,论文的创作过程使我深刻认识到了自己的不足,也对因时间和个人能力的局限未能在论文中提及的部分倍感遗憾。此文虽已结束,但科研的道路却远未结束。无论今后何处,在南开所学都会令我引以为傲。